\documentclass[11pt]{article}
\usepackage[margin=1in]{geometry}
\usepackage{graphicx}
\usepackage{pgfgantt}
\usepackage{hyperref}
\usepackage{fancyhdr}
\usepackage{graphicx}
\usepackage{hyperref}
\pagestyle{fancy}
\fancyhf{}
\rhead{Brian Tipton, Aiden Nelson, Austin Herkert}
\lhead{Final Project Team and Idea}
% \rfoot{Page \thepage}
\lfoot{\today}

\begin{document}

\begin{center}
	\LARGE \textbf{Race the Storm}
\end{center}

\section*{Game Overview}
Taking inspiration from the game \href{https://store.steampowered.com/app/253030/Race_The_Sun/}{\textit{Race the Sun}},
in which a geometric plane flies through generated polygon terrain in an attempt to stop the sun from setting, we are going to build a hovering 3D land speeder going through generated desert terrain.
It will be called \textbf{Race the Storm}.

\begin{figure}[h]
	\begin{center}
		\includegraphics[scale=0.1]{racethesun.jpg}
	\end{center}
	\caption{\textit{Race the Sun Gameplay Screenshot from \href{https://store.steampowered.com/app/253030/Race_The_Sun/}{Steam}}}
	\label{fig:racethesun}
\end{figure}

\subsection*{General Mechanics}
The user will be able to steer left and right through the shifting dunes and obstacles.
The more times that a user runs into an obstacle or goes without getting a boost, the more the impending sand storm catches up to them.
The speed only multiplies with each powerup you collect, ever so slightly increasing the difficulty of the game for each speed up.
Once the storm reaches them (from behind) or a player crashes, the level is completed and the player loses.

\subsection*{Aesthetic}
The overall aesthetic of the game will be of a arid desert landscape.
The players can expect to see pyramids, obelisks, and sphinxes containing powerups and points of interest.

\subsection*{Narrative}
The narrative is that the player controlling a hovering land speeder that is constantly speeding away from an impending sand storm that is sure to destroy it.
The only way to avoid that horrific fate is to race through the desert avoiding obstacles and collecting boosts to keep them moving faster and faster.

\pagebreak
\subsection*{Technology}
\begin{itemize}
	\item \textit{Unity}\\
	      We will be using Unity Game Engine as our development platform for rapid game protyping and testing.
	\item \textit{Assets }\\
	      Free and or attribution required assets will used to build the landscape for our game.
	\item \textit{Landscape Generation}\\
	      We will use a pseudorandom object generation algorithm for building the incoming environment as the player progresses.
\end{itemize}
\section*{Project Timeline}
\begin{ganttchart}[
		x unit=2.0cm,
		y unit chart=0.9cm,
		vgrid,
		bar/.style={fill=blue!50},
		incomplete/.style={fill=white},
		progress label text={},
		bar height=0.7,
		group right shift=0,
		group left shift=0,
		group top shift=.7,
		group height=.3,
		group peaks width={0.1},
		inline]{1}{9}
	\ganttgroup{Assets}{1}{2} \\
	\ganttbar{Selection}{1}{2} \\
	\ganttgroup{Camera}{2}{3} \\
	\ganttbar{Setup}{2}{3} \\
	\ganttgroup{Movement}{3}{5} \\
	\ganttbar{Implementation}{3}{4} \\
	\ganttbar{Testing}{4}{5} \\
	\ganttgroup{Obstacles}{5}{7} \\
	\ganttbar{Generation}{5}{6} \\
	\ganttbar{Testing}{6}{7} \\
	\ganttgroup{Powerups}{7}{8} \\
	\ganttbar{Implementation}{7}{8} \\
	\ganttbar{Testing}{8}{8} \\
	\ganttgroup{Goal}{8}{9} \\
	\ganttbar{Implementation}{8}{9} \\
	\ganttbar{Testing}{9}{9}
\end{ganttchart}
\end{document}
